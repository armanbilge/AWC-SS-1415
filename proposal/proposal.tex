\documentclass{article}
\usepackage[british]{babel}
\usepackage{csquotes}

\usepackage[backend=biber,style=alphabetic]{biblatex}
\addbibresource{\jobname.bib}

\usepackage{acronym}
\acrodef{AWC}{Allan Wilson Centre}
\acrodef{ESS}{estimated sample size}
\acrodef{HMC}{Hamiltonian Monte Carlo}
\acrodef{MCMC}{Markov chain Monte Carlo}
\acrodef{ODE}{ordinary differential equation}

\title{Implementing \acl{HMC} for Efficient Bayesian Evolutionary Analysis \\
           \Large\textsc{awc summer scholarship proposal}}
\author{Arman Bilge \\ \texttt{armanbilge@gmail.com}}
\date{29th September 2014}

\frenchspacing
\begin{document}

    \maketitle

    \paragraph{Introduction.}

    Today, the use of \ac{MCMC} methods is ubiquitous to software for the
        evolutionary analysis of molecular data.
    In particular, Bayesian statistical inference via \ac{MCMC} has transformed
        the field of phylogenetics by enabling the use of complex models for
        studying evolutionary processes, including molecular evolution,
        demographics, genome evolution, biogeography, speciation, and
        epidemics.
    However, the growing number of variables associated with these complex
        models increases the dimensionality of the complete model space,
        resulting in poor performance of the standard Metropolis--Hastings MCMC
        algorithm, which utilises a random walk to draw samples from the model
        space at a frequency relative to their posterior
        probability~\cite{Met+53,Has70}.
    As the number of dimensions increases, so does the number of steps, or
        perturbations to the model state, that are necessary to draw a
        pseudo-independent sample and often the rate at which proposed steps
        are rejected.
    An effort to improve on the widely-used \ac{MCMC} algorithm is well-aligned
        with the \acl{AWC}'s \emph{Imaging Evolution} initiative \enquote{to
        develop novel mathematical techniques and software for evolutionary
        genetic analysis} and will undoubtedly enhance the analysis process
        across several areas of biology.
    Here, I propose the evaluation of an alternative algorithm, \ac{HMC} (also
        referred to as hybrid Monte Carlo)~\cite{Nea11}, in the context of
        evolutionary analysis.

    \paragraph*{The \ac{HMC} algorithm.}

    \ac{HMC} is named for its use of Hamiltonian dynamics to deterministically
        traverse the model space and, as such, possesses a close analogy to a
        physical system.
    In the analogy, we liken the probability density of our current state to
        the potential energy of the system.
    Additionally, we augment our state with a momentum variable for every
        variable in our model, and we consider the probability density of these
        momentum variables (typically independent normal distributions) the
        kinetic energy of the system.
    Proposals are made by drawing random values for the momentum variables
        and simulating the dynamics for some fixed amount of time.

    Fundamentally, the \ac{HMC} algorithm may be thought of as an intelligent
        proposal mechanism, or operator, embedded within a standard \ac{MCMC}
        analysis.
    In fact, several properties of Hamiltonian dynamics, such as its
        reversibility, conservation of energy, and symplecticness, make it a
        particularly desirable operator \cite{Nea11}.
    For example, because conservation of energy is equivalent to the
        conservation of probability density by our analogy, it implies that the
        probability density of a proposed state equals that of the current
        such that, in theory, this operator will have an acceptance probability
        of one.
    Conveniently, multiple variables are updated simultaneously in a single
        Hamiltonian proposal; to achieve the equivalent under a standard
        \ac{MCMC} scheme requires the implementation of special operators.

    Simulation of these Hamiltonian dynamics relies on the solution to a system
        of coupled \acp{ODE}, where one of these equations is the partial
        derivative of the posterior density against every variable in the
        model.
    There are a number of numerical schemes available to approximate the
        solution~\cite{Nea11}.
    Note that because discrete spaces (e.g., tree topologies) are
        non-differentiable, the \ac{HMC} proposal strategy cannot operate on
        them.\footnote{It is possible to combine standard \ac{MCMC} updates for
        discrete variables with \ac{HMC}, although obviously the advantages of
        inference under \ac{HMC} do not apply to them.}

    While the \ac{HMC} algorithm has demonstrably better performance relative
        to standard \ac{MCMC} for specific examples \cite{Nea11}, it is
        unlikely that this will hold true in general.
    Therefore, it becomes important to evaluate the potential benefits of
        \ac{HMC} relative to \ac{MCMC} for particular types of problems and
        recognise the fine-tuning necessary to enjoy this enhanced performance.

    \paragraph*{The project.}

    The principal goal of this project is to demonstrate that an implementation
        of \ac{HMC} for Bayesian evolutionary analysis offers a practical
        advantage over existing \ac{MCMC} methods.
    Because of the challenges of sampling from discrete spaces, I will focus on
        the inference of parameters on fixed trees for a set of popular, core
        models, including DNA substitution models, the coalescent and
        birth--death models, and molecular clock models.
    The important considerations will be determining efficient methods to
        compute or approximate the partial derivatives of the posterior
        distribution with regard to the model variables, optimising the tuning
        parameters of the \ac{HMC} algorithm particularly for evolutionary
        analysis, and developing a modular implementation of the algorithm in
        an existing software package.
    To evaluate its efficiency against standard \ac{MCMC}, I will compare the
        \ac{ESS} per CPU-hour of my implementation against that of other
        popular softwares for Bayesian phylogenetic inference, i.e.,
        MrBayes~\cite{RH03} and BEAST~\cite{Dru+12,Bou+14}, under the same
        model.

    Should my \ac{HMC} implementation show promising results for inference on
        fixed topologies, the next steps will be to move towards a more
        complete implementation.
    Ultimately, an \ac{HMC} implementation for phylogenetics must be capable of
        efficiently traversing the treespace, particularly with respect to
        topology.
    Developing a method capapable of this would likely have an impact wider
        than simply the evolutionary biology community, as to-date there are no
        \ac{HMC} implementations that can sample these discrete spaces.

    \printbibliography

\end{document}
