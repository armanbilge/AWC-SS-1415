\documentclass{article}
\usepackage[british]{babel}
\usepackage{csquotes}

\usepackage[backend=biber,style=alphabetic]{biblatex}
\addbibresource{../hmc.bib}

\usepackage{acronym}
\acrodef{HMC}{Hamiltonian Monte Carlo}
\acrodef{MCMC}{Markov chain Monte Carlo}

\title{Implementing \acl{HMC} for Efficient Bayesian Evolutionary Analysis \\
           \Large\textsc{awc summer scholarship proposal}}
\author{Arman Bilge \\ \texttt{armanbilge@gmail.com}}
\date{29th September 2014}

\frenchspacing
\begin{document}

    \maketitle

    \paragraph{Introduction.}
    Today, the use of \ac{MCMC} methods is ubiquitous to software for the
        evolutionary analysis of molecular data.
    In particular, Bayesian statistical inference via \ac{MCMC} has transformed
        the field of phylogenetics by enabling the use of complex models for
        studying evolutionary processes, including molecular evolution,
        demographics, genome evolution, biogeography, speciation, and epidemic
        outbreaks.
    However, the substantial number of variables associated with these complex
        models dramatically increases the dimensionality of the overall
        model space, resulting in poor performance of the standard
        Metropolis--Hastings MCMC algorithm \cite{Met+53,Has70}.
    The \ac{MCMC} algorithm utilises a random walk to draw samples from the
        model space relative to their posterior probability.
    As the number of dimensions increases, so does the number of steps, or
        perturbations to the model, that are necessary to draw a
        pseudo-independent sample.
%   Furthermore, often the rate at which proposed steps are rejected (due to
%       low posterior probability) will also increase.
    Here, I propose the evaluation of an alternative algorithm, \ac{HMC} (also
        often referred to as Hybrid Monte Carlo), designed to address this
        shortcoming of \ac{MCMC} in the context of evolutionary analysis.

    \paragraph*{The \ac{HMC} algorithm.}

    While the \ac{HMC} algorithm has demonstrably better performance relative
        to standard \ac{MCMC} for specific examples, it is unlikely that this
        will hold true in general.
    Therefore, it becomes important to evaluate the potential benefits of
        \ac{HMC} versus \ac{MCMC} for particular types of problems.

    \paragraph*{The project.}
    The principal goal of this project is to demonstrate that an implementation
        of \ac{HMC} for Bayesian evolutionary analysis

    \paragraph*{Concluding remarks.}
    Should my \ac{HMC} implementation show promising results for inference on
        fixed topologies, the next steps will be to move towards a more
        complete implementation.
    Ultimately, an \ac{HMC} implementation for phylogenetics must be capable of
        efficiently traversing the treespace, particularly with respect to
        topology.
    Developing a method capapable of this would likely

    \printbibliography

\end{document}
