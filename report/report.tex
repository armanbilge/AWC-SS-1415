\documentclass{article}
\usepackage[british]{babel}
\usepackage{csquotes}

\usepackage[backend=biber,style=alphabetic]{biblatex}
\addbibresource{\jobname.bib}

\usepackage{acronym}
\acrodef{ESS}{effective sample size}
\acrodef{HMC}{Hamiltonian Monte Carlo}
\acrodef{MCMC}{Markov chain Monte Carlo}
\acrodef{ODE}{ordinary differential equation}

\usepackage{mathtools}
\newcommand{\dd}{\; \mathrm{d}}
\renewcommand{\vec}[1]{\ensuremath{\mathbf{#1}}}
\newcommand{\mat}[1]{\ensuremath{\mathbf{#1}}}

\title{Implementing \acl{HMC} for Efficient Bayesian Evolutionary Analysis \\
           \Large\textsc{awc summer scholarship report}}
\author{Arman Bilge \\ \texttt{armanbilge@gmail.com}}
\date{27th March 2015}

\frenchspacing
\begin{document}

    \maketitle

    \paragraph{Introduction.}



    Bayesian evolutionary analysis is centered around the posterior probability
        of a phylogenetic tree~$T$ given the molecular sequence
        data~$D$~\cite{Bou+14}, which is defined by
        \begin{equation}
            P\left(T \mid D\right)
                \propto \int_\theta P\left(D \mid T,\theta\right)
                P\left(T \mid \theta\right) P\left(\theta\right) \dd\theta
        \end{equation}
        where $\theta$ is a vector of nuisance parameters for the model.
    Because there is no closed-form expression for this multidimensional
        integral,


    As the amount of molecular data grows, as does the complexity of the models
        used to analyse them, the large dimensionality of the statespace makes
        evaluating this integral computationally intractable.

    \ac{HMC}, first described as hybrid Monte Carlo by \textcite{Dua+87},

    \paragraph{Methods.}

    Let $\pi\left(\vec{q}\right)$ be the target probability density.
    We augment the state space with momentum $\vec{p}$ and define the
        Hamiltonian for our system as
        \begin{equation}
            H\left(\vec{q},\vec{p}\right)
            = U\left(\vec{q}\right) + K\left(\vec{p}\right)
        \end{equation}
        with the potential energy
        $U\left(\vec{q}\right) = -\log{\pi\left(\vec{q}\right)}$ and kinetic
        energy $K\left(\vec{p}\right) = \frac{1}{2} \vec{p}^T \mat{M} \vec{p}$,
        where $\mat{M}$ is the mass matrix.
    To describe the \ac{HMC} algorithm, I introduce three operators:

    \paragraph{Results and discussion.}

    \printbibliography

\end{document}
